\chapter{绪论}
\section{研究背景与意义}
随着时代的进步,互联网、物联网、云计算、三网融合等IT与通信技术的迅猛发展,%
大数据成为信息技术领域的又一热门话题,在数据挖掘、人工智能、社会计算、生物学和化学等领域的应用日渐深入,%
信息社会已经进入了网络大数据时代。在当前网络信息时代的背景下,可以说大数据时代已经到来,%
而且这个新概念被赋予了极为丰富的内涵\upcite{李慕白2018网络大数据}。%
近几年,大数据越来越显示出巨大的影响作用,己逐渐应用于政治、医疗健康、学术科研、教育、图书馆服务等领域。
据统计,国外知名社交网站Facebook管理超过400亿张图片,所需的存储空间超过100PB,%
每天发布的消息高于60亿条,需10TB空间存储,Twitter每天产生1.9亿条微博;搜索引擎每天的%
日志超过35TB,Google搜索引擎一天需要处理的数据超过25PB。2010年全世界的信息总量是1ZB,%
最近3年人类产生的信息量已经超过历史上总信息之和,预计到2020年,%
总量将达到35ZB\upcite{张俊林2014大数据日知录,王元卓2013网络大数据}。%
大数据既给人们的生活带来了便利,同时也带来了巨大的挑战,该技术正在改变着%
人们的工作与生活方式\upcite{林海伦2017面向网络大数据的知识融合方法综述}。%

与此同时,自2007年起,国内外出现了很多专业的科研社交网站,如国外的ResearchGate、Academic%
国内有科研之友、学者网、百度学术等,为学者提供了线上的交流平台,%
科研交流有了新的有效途径\upcite{韩增义2011科技论文推荐系统研究与实现}。
在学术科研领域中,学术文献也以飞快的速度在增长,每天数以万计学术成果被发表,
1997年据R.M.May的统计,公开出版物的年增长速率为3.7\%,特别是在一些比较热门的研究领域,现在这
个数字更为惊人\upcite{陈海华2015学术文献引文推荐研究进展}。韩增义也在其论文中表示代表人类知%
识前沿的科技文献正在以每年6\%-8\%的速率增长,不同国家或地区之间合作论文的比例%
从2003年的13.2\%增加至2013年的19.2\%,2013年Scopus数据库收录的同行%
评议期刊所发表的论文数量达219.97万篇\upcite{韩增义2011科技论文推荐系统研究与实现}。%

面对如此庞大的数据,作为科研学者,要追踪所在学科的最新发展动向,%
就要追踪该领域的相关文献,分析这些文献中发表的新观点,用到的新技术,得到最新的研究热点以及新的成果,%
最后追踪到其研究突出的相关学者。然而,新兴技术往往是几个学科相互交叉的结果,%
而从一个学科去发现别的学科中出现的新技术、新主题或者相关学者就更加困难,%
如何快速有效地获取与自身研究相关的学者将是一个很有挑战的问题。%
目前,虽然已经有很多学者对社交网络开展了广泛研究,但多数是在探索该类网站对线上科研交流与%
学术创新的促进作用,特别是在国内,科研社交网络的研究大多是探讨国内外发展差距、调查用户需求等,%
国内科研社交网络的应用现状还远落后于国外,网站中的资源丰富性、内容可获得性、%
检索结果可用性等都差强人意。

本文的研究内容是结合自然语言处理和推荐系统相关理论知识,实现快速向用户推荐可能感兴趣的学者。%
充分利用文献摘要的重要性,提取文本的重要特征,构建推荐对象的兴趣模型,%
再结合用户搜索的关键词信息,构建相关模型来识别用户的搜索意图,综合考虑推荐对象和用户两者之间%
的特征信息挖掘两者之间的隐式联系,从而发掘出用户感兴趣的学者,
以此来帮助科研学者快速的找到有相同研究兴趣的学者。%
本文对科研社交网站中的学者推荐进行研究,有利于增强学术合作、提升科研人员学术交流、%
提高科研学者的工作效率和推动国内的科学研究都具有深远意义。

\section{国内外研究现状}
大数据时代,既给人们的生活带来了很多便利,同时也面临着很多的问题。你可以足不出户便轻松了解到所有%
的大小事情,同时大量的信息也阻碍我们获取高质量的信息。面对信息膨胀的问题,各个领域的研%
究者也都已经提出了解决办法,例如分类目录、信息检索和推荐,毫无疑问推荐系统在解决信息过载问题上是%
非常成功,自从推荐系统被提出之日起,便吸引了广大研究者的注意,而且被广泛的应用在各个领域中,%
例如电影推荐\upcite{bennett2007netflix,koren2009matrix}、%
音乐推荐\upcite{hicken2005music,cheng2014just,lee2017smartphone}、%
图书推荐\upcite{chien2017exploring}、广告推荐\upcite{kendall2018social}、%
电商推荐\upcite{guo2017application}以及学术领域推荐。本文的研究领域则是学术领域相关的推荐,%
科研社交网站作为一种新型的专业社交网络平台,主要关注的是为科研人员提供在线的以科学研究为导向的活动及构建学者%
间学术网络\upcite{熊回香2017科研社交网站中基于相似兴趣的学者推荐研究}。从2007年开始,国内外就出现%
一些科研社交相关的网站,如国内外的ResearchGate、Academic和科研之友和学者网等等,%
为科研交流有了新的有效途径。

经研究表明,在对文献数据进行挖掘时,论文正文篇幅较长,且包含的冗余信息%
较多,最能代表论文的内容的是论文的题目、摘要和关键词等部分\upcite{韩增义2011科技论文推荐系统研究与实现},%
关键词短小精悍,但其包含的信息却非常的丰富,
因此本文就采用文本挖掘和自然语言处理相关技术抽取出论文摘要中的信息作为关键词,以抽取出的关键词作为论文作者的兴趣。
在此基础上提出基于关键词网络的学者推荐模型,这里的关键词即是通过文本特征工程相关技术从文本中抽取出来的%
能代表作者研究兴趣的一些词或者短语;
学者推荐则是属于社交网络范畴,因此本文将从特征提取之关键词抽取和社交网络两个维度去阐述目前的研究现状。%

首先对关键词提取进行调研得知,关键词的提取是文本挖掘的一个子领域,而文本挖掘技术又是数据挖掘的一个分支,%
所以关键词抽取也就属于数据挖掘领域范畴。
1995年Feldman提出了文本挖掘概念和框架。国外研究比较早,在特征工程、文本分类等方面都取得丰富的研究成果。%
国内1998年开始才陆续开展文本挖掘的研究,并且由于中文自身的特点,难度系数也相对大,所以跟国外相比还存在%
着一定差距\upcite{杨阳2017文本挖掘技术在学术人物分析中的应用}。该技术在数据指数增长的时代扮演%
者重要的作用,很多文本挖掘工具都已经应用在商业方面\upcite{张俊伟2017基于语义分析的评论文本挖掘与商品推荐}。%
通过将非结构化的文本转化为结构化数据的形式,让计算机能够计算,从而抽取出隐含的、有用的知识。%
文本特征工程的技术包括预处理、特征提取等,数据的预处理对结果准确率有很大的影响,是一个很重要的环节。%
该项技术被广泛应用在文本分类、聚类和情感分析等各项研究中。%
其中文献\upcite{yang2018multi}就是采用将文本挖掘和深度学习结合对文本进行情感分析。%
50年代末期,国外学者.P.Luh就已经提出了词频统计的思想,用于自动分类,随后众多学者在该领域也取得卓越的成效,%
最近研究者主要围绕文本的挖掘模型、特征抽取和文本表示\upcite{梁楠2015基于文本挖掘的律师推荐方法研究与应用}。%
国内起步较晚,针对中文信息处理还未形成完整的技术理论和框架,不过进展也在逐步加快。



\section{研究内容和方法}
\subsection{研究内容}
本文通过调研分析科研社交网络的发展状况后,分析了目前科研社交网络在国内外理论和实际应用的情况,%
本文提出一种新的思路去解决学者推荐模型,详细阐述了该学者推荐模型的工作步骤和方式,%
受文本特征提取之关键词抽取的启发,本文还提出了结果评价的指标。最后本文通过爬取“微软学术”官网的%
真实数据对提出的推荐模型的有效性进行验证。本文总体分为六个章节,具体每章的内容安排如下所示:

\subsection{研究方法}
本文主要采用的研究方法有如下几点:
第一,文献调研法。通过互联网技术访问线上各个数据库中检索了大量研究领域的相关书籍、论文等学术成果%
,经过对国内外的相关研究文献与资料的全方位收集和分析,确立本文研究方向和主题,%
设计本文推荐模型框架及各模块之间的耦合。
第二,迁移法,本文是建立在自然语言处理、文本挖掘、推荐系统等相关技术的研究基础之上,通过综合%
探索以上技术理论,将其迁移到本文的模型框架及设计的评价指标上。
第三,实验仿真与分析法,本文通过调研和分析大量文献,提出本文的研究内容,为了验证本文提出的模型的%
有效性,本文在多个真实的数据上进行了实验,从而检验模型的可靠性。
\section{本文主要贡献}
本文通过大量前期调研,对比国内外科研社交网站在学者推荐技术上进行的研究,通过深入探索后提出了本文的%
研究问题,并针对提出的问题进行了大规模的对比实验,直到得出最后的结论。整个过程中本文的主要贡献体现%
出如下几点:
\begin{enumerate}
\item 本文提出了一个基于关键词网络的学者推荐模型,%
该模型能够根据新用户搜索的关键词进行及时推荐。%
该模型包含两个网络图,即关键词共现图(Keywords Co-occurrence Graph),%
为了描述方便,本文简称该图为$KCG$,$KCG$是根据用户学者检索的关键词和匹配到的文献的关键词%
共同构建的,其目的有二,第一是在用户学者没有明确的意图的情况下,通过挖掘$KCG$中的核心点作为%
用户学者的搜索意图,第二是通过该图确立用户学者的研究兴趣。另外一个是
论文学者(Author)和关键词(Keyword)构建的二部图(Graph),简称为$AKG$,该图的目的是%
采用某种算法对论文学者进行打分排序,以关键词为纽带,建立论文学者和用户学者之间的映射关系,%
从而向用户学者推荐最有可能感兴趣的论文学者。本文设计的推荐模型通过耦合并挖掘$KCG$和$AKG$两个网络图,%
最后完成最终的推荐目标。
\item 本文设计多层过滤器从每篇摘要中抽取关键词集,%
抽取出的关键词即为论文学者的研究领域或者兴趣。多层过滤器包括采用自然语言处理相关的Pos-Tag标注词性、%
过滤停用词、正则匹配等启发式的算法过滤生成关键词候选集,还包括文本特征提取中的多个经典%
特征算法对候选集做进一步过滤,最后采用聚合排序算法对候%
选关键词集进行重排序,从而生成高质量的关键词集。%
\item 在学者推荐领域,本文提出了全新的假设,即以关键词网络作为切入点,完成推荐任务。%
该假设是在缺乏其他指标,如行为记录、基本信息和论文引用数、影响因子等特征,%
仅仅只使用关键词这一个特征的条件下,设计出了本文的学者推荐模型,该模型能有效解决推荐系统%
的冷启动问题。为了验证有效性,本文爬取了微软学术官网的文献数据,在真实的数据上验证了模型的可行性。

\end{enumerate}




