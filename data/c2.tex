%!TEX root = ../document.tex
\chapter{个性化推荐、序列感知推荐及迁移学习}
\section{推荐系统}
推荐系统的定义和概念很多,但1997年Resnick和Varian给出的定义%
被广泛接受,既“推荐系统是利用电子商务网站向客户提供商品信息%
和建议,帮助用户决定应该购买什么产品,模拟销售人员帮助客户%
完成购买过程”。%

通用的推荐系统模型如图\ref{fig:frame}所示,图中%
可以清晰看到,%
推荐系统中很重要的3个模块分别是:用户建模模块、%
推荐对象建模模块推荐算法模块,它通过匹配用户模型%
中兴趣需求信息和推荐对象模型中的特征信息,%
同时结合使用相关的推荐算法进行筛选,推荐用户可能%
感兴趣的对象\upcite{王国霞}。
\begin{figure}[htbp] % use float package if you want it here
  \centering
  \includegraphics[width=\textwidth]{frame.jpg}
  \caption{推荐系统的通用模型}
  \label{fig:frame}
\end{figure}




\subsection{个性化推荐概述}
个性化推荐能成功,需要具备两个条件,第一是海量的信息,%
因为只有信息量很大的时候用户才需要系统的自动推荐。%
第二是用户没有明确的需求,如果有明确的需求,用户都会选择%
通过浏览器搜索,快速发现感兴趣的东西,而不需要推荐了。%
随着互联网行业的快速发展,信息量的增长速度飞快,%
各种新闻的推送(微博、微信公众号等),占用了很多的时间,%
严重的影响了获取信息的质量问题,%
大量的垃圾信息导致人们获取有价值的信息的成本有所增加,%
并且,人们的生活节奏也日益加快,需要在有限的时间内%
获取对自己有用的信息便难上加难,为了解决信息过载的难题,%
研究人员边通过用户历史行为数据开始对用户兴趣进行建模,从而%
实现个性化推荐的功能,让每个用户都有不一样的个性化页面。%
个性化推荐系统的价值便在于此。%
至今,国内外许多大型的公司投入了大量的精力到推荐系统的研究中,%
因此也给公司带来了很大收益,如最早研究推荐系统的亚马逊。%

在学术社交研究领域个性化人物推荐包括社交好友推荐,%
论文合作作者推荐等。收集用户的历史行为记录,如评分、%
分享、收藏、自定义标签等,通过分析用户的行为,%
给其推荐可能感兴趣的对象,以及用户的个人基本信息,%
如性别、年龄,在推荐系统的研究中,真正难以把握的是用户个性化需求,
\subsection{推荐算法}


\subsection{推荐系统存在的问题}

\textbf{冷启动问题}推荐系统需要数据作为支撑,通常需要根据用户历史行为记录去%
预测未来可能产生的行为和兴趣。现实中,我们面对大量的新用%
户又或者对于全新上线的系统不仅没有有效的用户行为数据而且%
也缺乏用户的个人基本信息(年龄、性别等),我们没有任何数%
据对用户的偏好进行建模,这个问题被称为“冷启动”,是推荐系%
统中面临的一个难题问题。在缺乏用户行为数据的情况下,%
并不是就没办法给用户推荐,
早期的推荐系统会基于商品内容数据做推荐,例如给商品打大%
量的标签或者通过推荐热门商品等方式来%
解决冷启动问题。Yao\upcite{yao2012product}等学者提出%
通过用户搜索的关键字信息去预测用户的偏好,%
具体方式是通过比较搜索的关键词和已有%
的产品的信息做相似性比较,实验证明,他们提出的方法比推荐%
最受欢迎的方法效果更好。

\textbf{噪音问题}大数据时代面临一个很严重的问题,不仅数据量多,而且数据不干净%
要获得高质量的数据往往是不太现实的,需要前期花费人力物力进行%
数据预处理,高效的数据处理能力,能快速的挖掘数据的重要信息。随着需求的提出,研究者们也提出了很多数据处理的%
方式,例如减噪、归一化等。

\textbf{数据长尾问题}个性化推荐系统一开始主要解决的问题就是通过发掘长尾数据来提升%
商家产品的销售额,而长尾商品却仅仅只能代表小部分用户的需求,%
所以只能充分发掘用户的行为,研究用户的兴趣,以此找到用户的个性化需求。%
所以,推荐系统可以更好地发掘数据的长尾。


\section{序列感知模型}
序列感知模型是把数据根据时间日期排序之后,按照数据之间的时间先后顺序,发现时间上近邻的数据之间的隐藏关系或数据的周期性变化规律等与时间有关系的一类数据挖掘模型,数据挖掘领域又常称这类模型为时序模型。因此序列感知模型面对的数据必须包含时间戳或者数据的存储形式能够不丢失数据的诞生先后顺序。时序模型数据分析的目的就是为了挖掘出数据之间的内在时间规律,找到这种时间规律之后利用其归纳、类推、演绎未来的数据变化趋势,从而进行建模样本之外的数据预测。


\subsection{序列感知推荐任务及场景}

当待分析的数据具有固有的顺序性质,序列学习方法就会在这些应用领域中有用,比较常见的应用有如自然语言处理、语音识别、时间序列预测、DNA建模,以及作为本文工作的核心内容,序列感知推荐。


\subsection{序列感知推荐技术分类}

\textbf{频繁集挖掘}

\textbf{马尔可夫链}

\textbf{循环神经网络}

因为传统前馈深度神经网络(FNN)无法了解给定输入的上下文环境关系,循环神经网络(RNN)\cite{RNN1994}被发明的目的就是用来进行对可变长度的序列数据进行建模。循环神经网络与传统的FNN模型之间的主要区别在于组成网络的单元中存在内部隐藏状态,在一个序列建模步骤中的每个内部隐藏状态节点都接收来自上一个节点的输入,因此这可以用一个循环来表示,其结构如图\ref{fig:rnn}所示,隐藏状态层保留了过去序列编码的摘要,每当RNN呈现新的输入时,就会更新该隐藏层的状态。对于一个最简单的标准循环神经网络其通过以下形式来更新隐藏单元的状态$h$:
\begin{equation}
  \label{eq:rnn}
  h_{n} = f \left(Ux_{n} + Wh_{n-1} + b \right)
\end{equation}
其中$h_{n-1}$是第$n-1$层神经网络的向量化表示,$x_{n}$是传递给第$n$层的输入序列编码,$U,W$是该层包含的权重矩阵,$b$是该层向量的偏置。函数$f(\cdot)$为非线性转换函数,也称激活函数,常用的激活函数有Logistic Sigmoid函数$\sigma(\cdot )$,$tanh(\cdot)$,线性整流单元$ReLU(\cdot)$和一些它们的变体。经过以上隐藏层状态的更新之后,输出层的计算公式如下:
\begin{equation}
  \label{eq:rnn1}
  \hat{y}_{n} = g(Vh_{n})
\end{equation}
其中$\hat{y}_{n}$是$n$时刻的输出值,$V$是输出层的权重矩阵,$g(\cdot)$是输出层的激活函数。

\begin{figure}[htb]%更改
  \centering
  \includegraphics[width=12cm]{rnn.png}\\
  \caption{简单循环神经网络结构图}
  \label{fig:rnn}
\end{figure}

通过公式\ref{eq:rnn},\ref{eq:rnn1}可以明白,RNN的每一个时间步骤都会有一个新的输入,并且特定时间步骤的输出依赖于之前所有步骤的输入,这意味着时间步骤$N$时刻的损失函数的计算要回溯到时间步骤$1$,这一过程也称为\textit{基于时间的反向传播算法(backpropagation through time, BPTT)}\upcite{Sutskever:2013:TRN:2604780}。但是如果要处理的序列很长的话,经过多层的反向传播,BPTT会产生梯度消息或者梯度爆炸的问题,以至于无法从差的很远的时间步骤中感知上下文环境,使得RNN的训练变得非常麻烦,RNN这一明显的缺点也称为长期依赖问题。

\textbf{长短期记忆神经网络}

为了解决RNN的长期依赖问题,一些基于公式\ref{eq:rnn}的变形工作诞生,其中最广为人知的是长短期记忆网络(long short-term memory, LSTM)\upcite{LSTM1997}。LSTM通过精巧设计的记忆单元更换了RNN中的隐藏单元,其核心计算单元如图\ref{fig:lstm}所示。LSTM神经元内部通过精心设计的分别称为遗忘门、输⼊门、输出门的三个门结构来决定哪些信息更新到内部或者从内部去除。在遗忘门\eqref{forget_gate}当中,前一时间步的掩藏状态和当前时间步的输入经过Sigmoid函数非线性转换之后得到一个$[0,1]$之间的值,以表示需要遗忘信息的概率。在输入门\eqref{input_gate},\eqref{input_gate1}中,将当前时间步骤中的输入和前一时间步骤学习到的隐藏状态经过tanh激活函数的计算生成一些候选值,并通过Sigmoid函数的传递从候选值中选出一些进行更新。而输出门\eqref{output_gate}就决定了当前单元要输出哪些部分。LSTM通过如下的组合函数来更新隐藏单元的状态:
  \begin{align} 
  f_{t} &=\sigma(U_{f} x_{t}+W_{f}[h_{t-1}+ c_{t-1}] +b_{f}) \label{forget_gate}\\
  i_{t} &=\sigma(U_{i} x_{t}+W_{i}[h_{t-1}+ c_{t-1}]+b_{i}) \label{input_gate}\\  
  c_{t} &=f_{t} c_{t-1}+i_{t} \tanh (U_{c} x_{t}+W_{c} h_{t-1}+b_{c}) \label{input_gate1}\\ 
  o_{t} &=\sigma(U_{o} x_{t}+W_{o}[h_{t-1}+ c_{t}]+b_{o}) \label{output_gate}\\ 
  h_{t} &=o_{t} \tanh (c_{t}) 
  \end{align}
% \end{equation}

其中$\sigma(\cdot)$是Logistic Sigmoid激活函数,而$i$、$f$、$o$和$c$分别是输入门、遗忘门、输出门和隐藏层的激活向量。

\begin{figure}[htb]
  \centering
  \includegraphics[width=12cm]{lstm.png}\\
  \caption{长短期记忆网络神经元结构图}
  \label{fig:lstm}
\end{figure}

\textbf{门控循环单元Gated Recurrent Unit}

由于LSTM在循环神经单元中增加了三个门结构,与RNN相比,在一个神经元当中要完成更多的复杂计算。当使用更大的网络的时候,训练时间相比RNN也将显著增加。为了减少训练的时间复杂度并同时保留LSTM对长期依赖关系的记忆能力,2014年Cho等人提出门控循环单元(Gated Recurrent Unit, GRU)\upcite{GRU2014}。与LSTM相似,GRU使用门结构建模单元内部信息的流动,不同的是,GRU将LSTM三个门减少为两个。GRU使用更新门来决定是否遗忘上时刻的信息或者记忆此时刻新的外部输入信息,其功能相当于组合了LSTM 中的输入门与遗忘门。使用重置门来决定如何将新的输入信息与内部已有放入记忆相结合。在$t$时刻GRU单元中更新门的状态表达式为:
$$
z_{t}^{j}=\sigma\left(U_{z} \mathbf{x}_{t}+W_{z} \mathbf{h}_{t-1}\right)^{j}
$$
其中,$\mathbf{x}_{t}$为第$t$个时间步骤的输入向量,$\mathbf{h}_{t-1}$中保存的是上一个时间步骤的信息。$U$,$W$是更新门当中输入向量与上一时间步信息的权重矩阵。$t$时刻GRU单元中重置门为:
$$
r_{t}^{j}=\sigma\left(U_{r} \mathbf{x}_{t}+W_{r} \mathbf{h}_{t-1}\right)^{j}
$$
结合更新门与重置门,整个GRU神经元在$t$时间步的内部状态更新为:
$$
z_{t}^{j}=\sigma\left(U_{z} \mathbf{x}_{t}+U_{z} \mathbf{h}_{t-1}\right)^{j}
$$
其中$\tilde{h}_{t}^{j}$为候选信息,表示当前记忆的内容,其计算表达式为:
$$
h_{t}^{j}=\left(1-z_{t}^{j}\right) h_{t-1}^{j}+z_{t}^{j} \tilde{h}_{t}^{j}
$$
单个GRU神经元的总体结构如图\ref{fig:gru}所示。

\begin{figure}[htb]
  \centering
  \includegraphics[width=12cm]{gru.png}\\
  \caption{门控循环单元神经元结构图}
  \label{fig:gru}
\end{figure}

\textbf{双向循环神经网络}

在自然语言处理的实体识别技术中,双向循环神经网络(Bidirectional recurrent neural networks, BRNN)\upcite{Schuster1997BidirectionalRN}弥补了单项循环神经网络对于上下文感知能力的不足,因为单向RNN预测下一个单词时使用的只是此单次出现之前的信息,而BRNN则从两个方向获取信息,上下文感知能力也就更强了。BRNN将隐藏层分为两个部分,前向状态层$\stackrel{\rightarrow}{h}$和反向状态层$\stackrel{\leftarrow}{h}$,其输出层的输入由$\stackrel{\rightarrow}{h}$和$\stackrel{\leftarrow}{h}$堆叠而成,其迭代公式如下:
\begin{align} 
  \vec{h}_{t} &= \sigma(U_{\vec{h}} x_{t}+W_{\vec{h} \vec{h}} \vec{h}_{t-1}+b_{\vec{h}}) \label{forword}\\
\stackrel{\leftarrow}{h}_{t} &= \sigma(U_{\stackrel{\leftarrow}{h}} x_{t}+W_{\stackrel{\leftarrow}{h} \stackrel{\leftarrow}{h}} \stackrel{\leftarrow}{h}_{t+1}+b_{\stackrel{\leftarrow}{h}}) \label{backword}\\
y_{t} &= W_{\vec{h} y} \vec{h}_{t}+W_{\stackrel{\leftarrow}{h}y} \stackrel{\leftarrow}{h}_{t}+b_{y} \label{stack}
  \end{align}

BRNN的隐藏层结构如图\ref{fig:brnn}所示。
\begin{figure}[htb]
  \centering
  \includegraphics[width=12cm]{brnn.png}\\
  \caption{双向循环网络神经结构图}
  \label{fig:brnn}
\end{figure}



\subsection{序列感知推荐算法评价指标}



\section{图神经网络}
学术社交网络能为具有共同兴趣的科研工作者提供%
一个实时沟通、共享成果的平台。目前,学术社%
交网络的发展很迅速,其覆盖的面也非常广泛,%
功能也逐渐强大起来,为科研工作者提供很多科%
研社交服务。随着推荐技术的成熟和发展,%
在系统中加入推荐功能成为社交网络的热点,%
可以通过潜在好友关系进行人物推荐,也可以根据相似兴趣%
推荐书籍或论文等。
实时为科研工作者推送一些推荐条目,不仅可%
以节约科研工作者的时间成本,好的用户体验,%
还可以吸引更多的新用户共享研究成果。有相关研究者对国外12个社交网络使用情况进行了调研,参加调研的人中%
超过3000科学家或工程师表示他们知道这些大型的设计网站,但是仅仅只有不到一半的人会定期的去访问ResearchGate。%
详细信息如图\ref{fig:research}所示,%
该图摘自2014年Richard Van Noorden在自然上发表的文章\upcite{van2014online},
从图中可以看出Google Scholar学术社交网站被定期访问的人数是最多的。针对学术社交网站的特殊性%
文中还对用户使用ResearchGate、Academia.edu和Mendeley三个社交平台的日常功能进行了调研,%
详细如下图\ref{fig:three}所示。
国内的社交网站也非常多,例如微博、微信、知乎、百度学术、科研之友、学者网等,但是,本文暂时没有检索到类似的统计相数据。
\begin{figure}[htbp] % use float package if you want it here
  \centering
  \includegraphics[width=\textwidth]{research.png}
  \caption{国外社交网络使用情况}
  \label{fig:research}
  \footnotesize
  注解:深蓝色表示知道有社交网站并且会定期访问的占比,浅蓝色表示知道有这些社交网站但是不会定期访问%
  的占比,黄色表示不知道有这些科研网站的比例。
\end{figure}
\begin{figure}[htbp] % use float package if you want it here
  \centering
  \includegraphics[width=\textwidth]{three.png}
  \caption{调研定期访问学术科研网站的学者的目的}
  \label{fig:three}
  \footnotesize
  注解:1表示不常用,2表示出于好奇,3表示可能有学者联系,4表示追踪指标,5表示找工作,6表示找合作伙伴,%
  7表示查看推荐的论文,8表示联系同行,9表示发布研究内容,10表示分享链接,11表示发起相关研究的讨论,%
  12表示对研究进行评论,13表示跟踪讨论进展;其中有1589位常访问ResearchGate的用户参加了调研,283位%
  参加了Academia.edu调研,198位参加了Mendeley的调研。
\end{figure}

社交网络的发展时间脉络如图,引自\ref{杨雪萍2017科研社交网站中的学者推荐研究}
\begin{figure}[htbp] % use float package if you want it here
  \centering
  \includegraphics[width=\textwidth]{timeline.jpg}
  \caption{社交网络的发展}
  \label{fig:academic_time}
\end{figure}

为了对比国内外学术社交网络发展的状况,
接下来,本文着重介绍国内外四个知名的科研社交网络平台:%

\textbf{ResearchGate}\footnote{\url{https://www.researchgate.net}}被称为“科学研究的脸书%
(facebook for research)”,%
自2008年上线到现在,该平台的研究者数量已经高达15亿多,包括45名诺贝尔获得者,
学者可以在此网站上分享研究成果、学术论著、并且也可以%
加入一些科研论坛小组。通过注册时填写感兴趣的领域、%
专业知识等信息,其他跟你有相似研究兴趣的科研工作%
者可以很容易发现你。根据个人兴趣,ResearchGate会推荐有相似研究兴趣的研究成果,%
此外,该网站还会根据用户兴趣推荐工作,%

\textbf{Academia.edu}\footnote{\url{https://www.Academia.edu}}于2008创建,截止到2018年12月,%
该网站统计已经有71亿学者加入Academia.edu科研社交网站中,上传了21亿论文,并且每月的访客%
超过40亿,是一个非常庞大的数字。发表在PLOS
ONE上的一篇文章表示,上传到Academia.edu的论文的5年内的引用率可以提高原来的69\%。


\textbf{学者网}\footnote{\url{http://www.scholat.com/}}也%


\textbf{Microsoft Academic}\footnote{\url{https://academic.microsoft.com}}
\textbf{Google Scholar}\footnote{\url{https://academic.microsoft.com}}
\textbf{百度学术}\footnote{\url{https://academic.microsoft.com}}

\subsubsection{迁移学习研究内容}


\subsubsection{迁移学习分类}

\subsubsection{迁移学习方法}

\subsubsection{迁移学习面临的问题}

\section{本章小结}

