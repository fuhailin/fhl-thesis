%!TEX root = ../document.tex
\begin{cabstract}

    随着互联网信息的不断增长,推荐系统已成为克服信息过载的有效策略。推荐系统作为帮助%
    人们从浩如烟海的信息中发现自己所需的重要工具,在许多网络应用中被广泛集成,不仅%
    辅助人们做出选择还帮助服务商提升了交易量,推荐系统的效用不容小觑,已成为信息社会%
    日常生活的重要组成部分。推荐算法作为推荐系统背后的核心力量,支撑着推荐系统的表现。%
    因此研究泛化能力强鲁棒性好的推荐算法一直是数据挖掘领域的热点。
    在该领域内的研究通常基于构建用户与物品的交互矩阵,这样的话对于每个用户-项目对,仅%
    能考虑一个交互行为(例如,评分)。
    但是,随着时间的推移,可以记录不同类型的多个用户-项目交互。
    并且在现代推荐算法中,大多数方法都忽略了用户所消费物品之间的时序关系。

    近年来,深度学习在计算机视觉和自然语言处理等许多研究领域引起了相当大的兴趣,这不%
    仅源于其出色的大数据计算能力,还归功于其能从原始数据学习特征表达的迷人特性,这一%
    影响早已扩散到了信息检索和推荐系统的研究领域。因此本文引入在自然语言处理领域表现
    出色的序列建模算法,创新地引入推荐系统领域,提出了两种利用神经网络技术挖掘用户%
    行为时序特征进行序列预测的新颖推荐算法,并从推荐结果精确性和计算效率方面进行了%
    分析,为现有推荐算法遇到的问题提出了可行地解决思路。

    为了验证本文提出模型的性能,分别在真实的用户行为数据集Movielens,Yoochoose上%
    进行了实验,实验结果表明,基于双向长短期记忆神经网络和自注意力机制的序列推荐模型,面对稀疏、大%
    规模数据时,也能较好地反映用户的短期兴趣。
    % 迁移学习应用在推荐领域也能利用辅助信息提升推荐准确度。

\end{cabstract}
\ckeywords{推荐系统;序列建模;循环神经网络;注意力机制;数据挖掘}

\begin{eabstract}

\end{eabstract}
\ekeywords{Recommendation system; Recurrent neural network;Transfer learning; Data mining;Cold start}

