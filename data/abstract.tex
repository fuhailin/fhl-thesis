\begin{cabstract}
互联网技术迅速发展,各行各业的数据都在指数膨胀,科研领域也一样,数据量急速的增长给科研人员带来%
了丰富的信息,同时也带来了许多难题。作为科研学者,需要实时保持追踪学术的新动态,然而,通过人工%
从海量数据中筛选出与自己研究领域相符的学者,无疑是一个非常浩大的工程。因此,本文使用自然语言处%
理方面的技术快速提取文本的关键词,再结合图论相关算法,快速筛选出与用户学者研究兴趣相符的科研学者,%
进而推荐给学者用户,以帮助科研人员节省时间,减小工作量。%

本文提出了一个基于关键词的学者推荐模型,该模型包含挖掘论文学者研究领域(关键词提取)、用户学者兴趣%
(关键词网络)和打分推荐三个子模块。主要流程是通过融合多种特征从论文摘要中提取出关键词集合,将关键%
词集作为论文学者研究领域;接着,将关键词集和用户学者搜索的关键词结合,构建以共现为关系的词图KOG,%
通过多种算法挖掘KOG图得到该图的核心节点,本文将得到的核心节点作为对用户学者偏好或者兴趣的一个扩展;%
最后,构建论文作者(Author)-关键词(Keyword)的二部图AKG,来对论文学者用户打分排序,%
其中关键词出现在论文学者的摘要中,作者节点和关键词节点之间就存在边相连。本文充分利用关键词%
去挖掘用户学者-论文学者之间的隐式联系。%

本文之所以关键词作为切入点,是因为短小精悍的关键词不仅能够代表一篇论文的主题,而且其本身%
蕴含着丰富的信息,其被广泛的应用在文本分类、聚类、搜索、推荐等领域。本文只考虑使用关键词%
(不使用任何其它的特征,例如论文引用数、作者影响力、用户学者的基本信息或者历史记录等)实%
现学者推荐,所以本文对于解决推荐系统中的冷启动问题有很重要的作用。无论是从用户学者的角度还是%
从论文学者的角度考虑,本模型都围绕关键词展开,关键词是影响该学者推荐模型准确率的唯一因素,%
因此,关键词抽取问题即是本文研究的核心部分,本文提出了融合多特征的关键词抽取算法,并对结果进行评估。%

本文在KDD、WWW真实的论文数据集进行关键词提取的对比实验。为了验证本文推荐框架的有效性,%
本文爬取了Microsoft
Academic官网的生物医学论文数据进行实验验证,实验证明,该推荐模型到不错的准确率,%
能进行有效的推荐,对于解决大数据时代信息过载问题,本文的研究具有重要的实际意义。

\end{cabstract}
\ckeywords{学者推荐;关键词抽取;词网络挖掘;核心点选取;冷启动}

\begin{eabstract}
  
\end{eabstract}
\ekeywords{scholar recommendation; extract keyword; cold start }

