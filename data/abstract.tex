\begin{cabstract}

    随着互联网信息的不断增长,推荐系统已成为克服信息过载的有效策略。推荐系统作为帮助人们从浩如烟海的信息中发现自己所需的重要工具,在许多网络应用中被广泛集成,不仅辅助人们做出选择还帮助服务商提升了交易量,推荐系统的效用不容小觑,已成为信息社会日常生活的重要组成部分。推荐算法作为推荐系统背后的核心力量,支撑着推荐系统的表现。因此研究泛化能力强鲁棒性好的推荐算法一直是数据挖掘领域的热点。

    近年来,深度学习在计算机视觉和自然语言处理等许多研究领域引起了相当大的兴趣,这不仅源于其出色的大数据计算能力,还归功于其能从原始数据学习特征表达的迷人特性,这一影响早已扩散到了信息检索和推荐系统的研究领域。在现代推荐算法中,大多数方法都忽略了用户所消费物品之间的时序关系,并且对于缺少数据的用户,现代推荐系统也不能给出精准的结果。	所以本文基于双向长短期记忆神经网络提出了一个新颖的时序推荐模型,它通过捕捉用户消费物品的时序特征来预测用户下一个的兴趣物品会是哪一个。借用迁移学习的能力来处理能启动问题。

    为了验证这一模型的性能,本文分别在真实的用户行为数据集Movielens,Last.FM上进行了实验,实验结果表明,基于双向长短期记忆神经网络的序列推荐模型,面对稀疏、大规模数据时,也能较好地反映用户的短期兴趣;迁移学习应用在推荐领域也能利用辅助信息提升推荐准确度。

\end{cabstract}
\ckeywords{推荐系统;循环神经网络;迁移学习;数据挖掘;冷启动}

\begin{eabstract}

\end{eabstract}
\ekeywords{Recommendation system; Recurrent neural network;Transfer learning; Data mining;Cold start}

