%!TEX root = ../document.tex
\chapter{实验结果与分析}

\section{数据集介绍}

为了比较上文提出的两个算法模型,分别从推荐结果的准确性、和算法训练与结果查询返回效率两个方面分别进行了两组实验。%
因此需要设计两组实验分别从算法推荐准确性和性能两个方面进行验证。本文评估上述两个算法在两个来自真实%
应用上的数据集,这些数据集都包涵用户与物品产生交互的时间戳:

\textbf{MovieLens}:MovieLens\footnote{\url{https://grouplens.org/datasets/movielens/}}是%
用来对推荐模型进行评估的最流行的基准数据集,它是由GroupLens研究组织从MovieLens网站收集的关于用户%
对电影评分的数据集,其中主要的评分文件以“用户ID | 电影ID | 评分 | 时间戳”为一行的格式%
保存了某个用户在某时刻对某个电影做出的评分标准,按照数据量的大小不同,本文选用的MovieLens数据集有%
Movielens 100K和Movielens 1M。

\textbf{Yoochoose}:Yoochoose\footnote{\url{http://2015.recsyschallenge.com}}数据集是%
2015年推荐系统顶级会议RecSys举办的挑战赛RecSys Challenge 2015上公开使用的目标数据集。%
Yoochoose包含了一个在线电子商务网站在6个月时间内用户所有的点击会话流。%





\section{评价指标}
作为一个推荐模型,模型会对查询目标给出点击可能性最高的N个物品,对于本文提出的序列推荐形式,%
模型的输入是$t$时刻及其以前的用户点击物品序列,需要的模型输出目标是$t+1$时刻用户可能点击%
的物品,以$\hat{I}_{u}^{t+1}$表示,所以本文采用了$Precision@N$, $Recall@N$指标来评估%
我们的模型,其计算形式如下:

\begin{equation}
Precision@N=\frac{\sum_{u}|\hat{I}_{u}^{t+1}\cap I_{u}^{t+1} |}{\left | \mathbb{U} \right |*N}
\end{equation}

\begin{equation}
Recall@N=\frac{\sum_{u}|\hat{I}_{u}^{t+1}\cap I_{u}^{t+1} |}{\left |\sum_{u}|I_{u}^{t+1}| \right |}
\end{equation}

由于$Precision@N$和$Recall@N$值可以通过阈值调整来相互权衡,为了得到一个更加容易量化评估%
的指标,本文引入了$F1@N$指标,如果$F1@N$分数越大,可以认为模型的效果更好。而当一次为%
用户推荐多个结果时,推荐结果在屏幕上展示的位置也会影响物品被点击的概率,越靠前的物品越可能被点击,%
约强相关的推荐物品应该出现在结果列表的越前面,而引入$NDCG@N$指标的原因就是为了感知这种%
位置的影响,让点击率越高的物品位置越靠前,$NDCG@N$%
值越接近1,得到的相关推荐结果中前N个物品的排序越准确,所以$F1@N$和$NDCG@N$的计算形式如下:
\begin{equation}
F1@N=\frac{2\times Precision@N\times Recall@N}{Precision@N+Recall@N}
\end{equation}
\begin{equation}
  \begin{aligned}
  DCG@N &= \sum_{i=1}^{N} \frac{2^{rel_{i}}-1}{\log _{2}(i+1)} \\
  IDCG@N &= \sum_{i=1}^{|REL|} \frac{2^{rel_{i}}-1}{\log _{2}(i+1)} \\
  NDCG@N &= \frac{DCG@N}{IDCG@N}
  \label{E21}
  \end{aligned}
\end{equation}
当推荐的物品是用户点击的内容时$DCG@N$中$rel$取值为1,否则$rel$取值为0。而由于每条样本仅有一条为相关,其余均为不相关,所以 $IDCG@N$ 对于每条样本都是一样的,即$\frac{1}{\log 2}$。

\section{实验环境}

% Please add the following required packages to your document preamble:

\begin{table}[H]
\centering
\caption{实验软硬件环境配置表}
\label{tab:set}
  \begin{tabular}{@{}ll@{}}
  \toprule
  实验环境 & 环境配置                                  \\ \midrule
  CPU  & Intel(R) Xeon(R) CPU E5-2650 v3 @ 2.30GHz *2 \\
  GPU  & Tesla K80 *4                                 \\
  内存   & 64126MB                                      \\
  OS   & CentOS Linux 7 (Core)                        \\
  编程语言 & Python 3.6.8                                 \\
  开发工具 & TensorFlow-GPU 1.12.0                        \\ \bottomrule
  \end{tabular}
\end{table}

\section{实验结果与分析}


\subsection{关键词抽取结果}



\subsection{学者推荐结果}




\subsection{结果分析}

\section{本章小结}



