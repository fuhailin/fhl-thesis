%!TEX root = ../document.tex
\chapter{基于双向长短期记忆网络的序列感知推荐}
学术社交网络的目标是建立一个能发现用户兴趣的模型,%
一方面用作者已发表的论文的摘要去探索作者的研究领域,%
将具有共同兴趣的人联系起来,这样既可以提高已有用户%
的忠诚度,又有助于吸引新的用户,以此提升网络的流量。%
另一方面通过挖掘关键词共现网络,获取用户学者的潜在研究兴趣,%
匹配两者的兴趣点,将有相似潜研究兴趣的论文作者推%
荐给用户学者。准确的推荐功能不仅能提高用户的%
参与度,同样也可以提高社交网络的影响力。%
在社交网络中,通常会倾向于将有相似兴趣的人建立联系,%
因此对人物关系和兴趣定位建模是两个很重要的部分。%
本文构建关键词网络和论文作者-关键词网络,%
并将两人物关系网结合起来,可以更好的预测出%
与用户学者更相似的论文作者,从而使本文的推荐性能更好。%

\section{基于双向长短期记忆网络的序列感知推荐模型}
\subsection{目标问题定义}

本文设计的学者推荐模型的目标是给无任何历史记录和%
个人资料的情况下,仅仅获取用户学者检索时输入的关键词,在文献数据的基本上实现学者推荐,%
该文献数据包括论文的标题、论文的摘要、论文的作者三个字段。通过用户检索关键词建立用户学者和%
论文作者之间的隐式联系,根据发现的联系给用户学者推荐其可能感兴趣的论文作者。%
举个简单的应用场景,对于一个新服务应用上线后,%
除了能搜集到用户的关键词外,基本无法获得一个游客用户的任何信息,针对游客怎么进行推荐?%
这个问题在推荐系统中被称为冷启动问题,本文以该问题为切入点,进行深入的研究,提出了下文%
的学者推荐模型,该模型在解决冷启动问题上非常优秀。%
并且该模型具有很强的移植性,能够在各式各样的场景中应用,例如将该模型应用在学术科研领域,%
不仅可以给学者用户带来很多好处,同时也可以给系统研发方本身也带来一定的益处。%
对于学者而言,该模型能够以少量的信息发挥无限的可能,快速的及时帮助用户%
学者挖掘到其感兴趣的领域学者。
对于系统而言,
该模型成功应用在科研社交应用中能够提高用户学者的用户体验,从而吸引旧用户再次回%
访或者新的用户注册成为长期用户,以达到提高用户的留存率、网站点击率及其它效果。

关键词的高度抽象既是用户和作者的研究兴趣,%
因此,本推荐模型通过挖掘用户学者搜索的关键词和作者发表%
的论文摘要,找到该用户学者和论文学者之间的潜在联系。%
通过耦合关键词网络($AKG$)和论文作者-关键词网络($AKG$),%
发掘用户学者和论文学者的共同兴趣。%

\subsection{模型介绍}



\section{关键词网络}
文档由许多词语特征组成,关键词对文档具有一定表征能力,%
因此,从与用户检索匹配到的文献摘要中抽取出关键词,这些关键词%
在一定意义上用户学者检索的内容,另一角度也反应了论文学%
者的研究内容,因此,本文的关键词网络是基于词项共现(Term Co-ocurrence)构建的,核心思想是用词项之间的共现程度%
反映语义之间的联系,经过多种方式挖掘关键词网络,获得%
将用户学者和论文作者之间联系较为紧密的词项,以此得到%
了两者之间联系的桥梁。

在没有推荐系统的时期,一个科研学者想投身于一个全新的研%
究领域时,一开始他只简单的了解了该领域的概况,对于该领%
域的具体分支以及对哪个分支的研究内容感兴趣,因此,他会%
通过关键词检索,检索并查阅大量的文献,从而确定自己的研%
究内容。例如张三想研究自然语言处理(NLP)领域,但自然语言处理包含很多任务,如文本内部特征研究(词性标注、分词等),文本分类、聚类,%
特征提取(命名实体识别、关键词抽取),知识图谱应用等。%
首先他可能会检索“自然语言处理”,得到系列该领域的文献概述,%
通过大量阅读获取到各个分支领域的信息,张三感觉关键词抽取、%
命名实体识别两个话题比较感兴趣,他则会依次检索“关键词抽取”、%
“命名实习识别”获取相关的信息(如该领域研究现状、%
出色的学者及相关的文献)。总之,科研人员会通过反复检索%
关键词获取目标信息,以此开展后续的研究。因此本文将用户%
搜索的关键词与该关键词匹配的文献关键词进行关键词网络构建,%
该网络的功能主要是用来检测用户学者的搜索兴趣和明确检索意图,%
通过基于图的算法挖掘关键词网络,自动预测他可能会感兴趣的研%
究内容,进而也为后续给他推荐相似的学者奠定基础。

\subsection{构建关键词网络KCG}
构建截图,关键词网络的作用,数据节点和边数表%

本节主要介绍关键词网络$KCG$的构建过程,根据用户搜索的关键词,系统会匹配数据库里的数据,%
将包含搜索关键词的文献记录返回,通过抽取出文献的关键词,将关键词以共现的关系构建出关键词网络图%
,即$KCG$网络图。例如用户搜索关键词为$"HFMD"$,返回包含$"HFMD"$的文献摘要,动态抽取出每篇文献摘要的%
关键词,再将这些关键词以共现关系构建成$KCG$图(为了加快计算过程,实现快速推进的效果,本文实现先离线抽取出每篇文%
献摘要关键词,用户搜索后直接返回关键词集合),共现窗口大小$window$为摘要的长度。%
下表\ref{tab:KCG}为在四个数据集上分别构建的词共现图的节点数目和边数目详细信息,在手足口病数据集上的节点数为%
643,边数为7705,在癌症数据集上的节点数为164325,边数为4223234。可见在癌症的$KCG$网络图非常大。%



\subsection{挖掘KCG网络核心点}



\section{学者-关键词模型}

本文将学者推荐问题建模在二分图上,该二分图网络的节点包括被推荐的对象,即论文%
作者($Author\_node$),和论文的关键词($Keyword\_node$),如果$Keyword_i$出%
现在作者$Author_i$的论文中,则$Keyword_i$和$Author_i$存在一条边相连,%
该图是一个无向无权图,为了描述方便,本文将该二分图简称为AKG。通常情况下,推荐系统是在寻找用户%
Users集合中每个$user_i$和被推荐对象Authors集合中的$author_j$之间的关系,%
例如可以通过计算的方式给每对($user_i$,$author_j$)组合进行打分,这个分数就可以被用来衡量用户$user_i$%
可能对被推荐对象$author_j$感兴趣的程度,本文用兴趣即关键词作为联%
系$user_i$和$author_j$两者的桥梁。本文尝试采用PersonalRank算法对构建的AKG二分图中的节
点进行打分,最后将Top K个得分较高的论文作者作为推荐对象推荐给用户。%

\subsection{构建学者-关键词AKG网络}
通过上段的描述可知该$AKG$网络图的构建方式,用邻接矩阵$M=(M_{i,j})$来表示$AKG$的权值和边,%
即$M$定义如下~(\ref{equ:chap03:M}),本文通过模拟用户检索关键词的过程,%

\subsection{Top K推荐打分算法}
为了发现被推荐对象与用户之间关系,目前已经有很多文献提出的算法是基于图结构的数据上,%
例如Gori发表在IJCAJ会议上的一篇关于电影推荐的论文\upcite{gori2007itemrank},%
Gori将电影推荐问题建模在图上,并且改进PageRank算法后提出ItemRank,采用ItemRank%
算法对图中的节点进行打分,本文借鉴其思想,采用PersonalRank算法对AKG网络图进行打分,%
最后给用户推荐分值高的前Top K个可能感兴趣的论文作者。%
PersonalRank\upcite{haveliwala2002topic}算法于2002年被Haveliwala提出,%
是一种基于随机游走的图算法,该算法与经典网页重要性排序PageRank算法非常相似,%
为网络中的每个节点计算重要性得分,并且文献\upcite{wu2018ga}表示其在网络图中表现出很好的性能。%
PersonalRank算法的计算公式如下~\ref{equ:chap03:pr}所示:
\begin{equation}
\label{equ:chap03:pr}
  \mathbf{PR(i)}=\frac{(1-\alpha)}{r_i}+\alpha\sum_{j\in
  in(i)}\frac{PR(j)}{|out(i)|}\quad
   r_i =
  \begin{cases}
   1\quad i=u \\
   0\quad i!=u
  \end{cases}
\end{equation}
\section{本章小结}



