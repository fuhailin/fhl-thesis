\chapter{基于聚类的关键词抽取}
关键词抽取任务通常包括三个基本步骤,实验的方法多种多样,可基于句子抽取,也可以基于内容理解的抽取,又或者是基于%
结构的抽取方法等等,无论什么方法被采用,基本上都会面临三个重要的问题,其一是文档冗余信息的识别及处理;%
其二是对核心关键词的定位识别;其三则是生成的词或短语的可读性和连贯性。宗成庆老%
师\upcite{宗成庆2013统计自然语言处理}对识别冗余信息的方法总结为两种%
.
\section{数据预处理}
对每个数据集,本文都使用以下的数据预处理步骤:%
首先对文本进行句子切割、分词和词性标注。对候选短语的打分采用如下策略,如公式~(\ref{equ:chap04:phrase})
\begin{equation}
\label{equ:chap04:phrase}
PhraseScore(p_i) = \sum\limits_{{w_j}\in {p_i}} WordScore(w_j)
\end{equation}

\section{对比算法}

\textbf{TFIDF}(term frequency-inverse %
document frequency)是基于统计的算法,对论文的摘要进行关键词抽取,%
TF是指该词在文本中出现的频率,%
可以用来描述文档内容;IDF是该词的逆文档频率,%
是用来衡量该词区分分档的能力。TFIDF的为两个公式的乘积,如~(\ref{equ:chap04:tfidf})所示。

\begin{equation}
\label{equ:chap04:tfidf}
TFIDF = {\frac{n_i}{N_j}*\log\frac{|D|}{|D_i| + 1}}
\end{equation}

其中N表示文档j中包含的词的总量,n表示词i出现在文档j中的词频。%
$|D|$表示语料库中总的文档数量,$|D_i|$表示语料库中包%
含词i的文档数量。%

\textbf{TextRank}基于词共现的特征-TextRank

\textbf{PositionRank}基于位置的特征-Position PageRank

\textbf{TopicRank}本文是基于TopicRank的一个改进算法,%
首先通过将候选集中的短语通过聚类划分为主题,即主题便是由%
候选集中的短语的子集表示,
再将聚类后得到的主题作为构建完全图的节点,该图的边则是通过计算不同主题之间词与词%
之间的语义关系得到图的权重,%
最后采用基于随机游走的PageRank算法为每个主题的排序,最后从每个主题中选出一个候选词%
作为关键词。该特征的获取需要进行的计算包括主题聚类、构建图和PageRank给节点打分,%
因此,相比较前面几种算法而言,该算法的速度相对缓慢。接下来将着重介绍TopicRank%
识别和定义主题的原理及本文对其进行改进的点。%

\section{关键词抽取聚合排序算法}


\subsection{算法描述}
\subsection{算法实现}
Borda Count
Schulze方法
Weighted Majority Voting


\section{本章小结}



