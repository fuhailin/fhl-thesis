\begin{ack}
时光飞逝,犹如白驹过隙。不知不觉,已在华南师范大学度过了求学生涯中又一个重要的阶段。回望这三年的心路历程,%
既有坎坷不平又不乏欢乐愉悦,回想起来都历历在目。在大家的帮助下,我得到的成长和进步,在此感谢所有人!

首先要感谢的是我的导师朱佳教授。刚进入华师时,我懂的甚少,自信心不足,也不会主动去探索发现新鲜事物,跟大多数学生%
的学习方法类似,整天抱着一本基础书本在啃。导师观察了我的学习方式后并没有直接告诉我这样做是错的,而是让我尝试%
亲自动手实践起来,就这样,导师的指导方式渐渐地融入我的日常学习生活,他总是在不知不觉中指导我摒弃掉我身上的不足,%
在这里,我学会了思考,增长了见识,学到了新本领,能独自应对和解决问题。非常感谢导师带给我的迅速成长,学生铭记于心。

感谢团队里的所有老师。感谢肖菁老师、曹阳老师和黄晋老师开题以来给我的毕业论文提出的宝贵意见,在你们的不断提点下%
,我的毕业论文才得以顺利完成。

感谢实验室的伙伴们,正是你们日夜的陪伴,让我的研究生生活丰富多彩。特别感谢我的大师兄许传华、二师兄武兴成,%
无论遇到什么难题,只要与你们交谈,都会给我指出多条可探索的门路,总是在最迷茫的时候给我指点迷津。%
感谢郑泽涛、伦家琪、胡迎彬、于晗宇、余伟浩、陈善轩、章婷华、于曼丽等师弟师妹,与优秀的你们在一起,也总能%
让自己也变得优秀一点点,最后祝愿你们的论文越发越好越多。

感谢我的家人在背后默默的支持和鼓励。求学的地方与家相距千里,平日我们都只能视频里简明,%
只有春节的时候才会回家与你们相聚几周,但我们的关系依然很亲密,是你们塑造了我,希望我以后能成为你们的骄傲。

感谢论文评审和答辩的专家教授们,感谢你们百忙之中来参与我的答辩,并且给我的论文提出了很宝贵的意见。

\end{ack}
