%!TEX root = ../document.tex
\chapter{总结与展望}
\label{chap:conclusion}

\section{论文总结}
推荐系统是数据挖掘与机器学习领域中最成功的应用之一,推荐算法作为推荐系统的灵魂,支撑着系统的表现。经典推荐算法面对大数据的压力时往往显得力不从心,而基于梯度下降优化的深度学习推荐算法没有理论上大数据上限。现有的推荐算法又往往忽视了隐藏在用户行为之中的时序关系,目前大多数深度学习推荐算法又需要大量的特征,当面对缺少用户隐私数据的场景这些算法都将无能为力。为了应对当前推荐算法这几个方面面临的问题,本文完成的工作如下:
\begin{enumerate}
    \item 本文调研了深度学习在推荐系统应用领域的发展情况,针对现有推荐算法都需要使用大量用户隐私%
          数据构造特征的情况下,提出了只利用用户历史行为序列的序列感知推荐算法。
    \item 本文设计了基于双向长短期记忆网络的序列感知推荐模型BiLSTM4Rec,对用户历史行为序列进行嵌入式表达,生成稠密矩阵,%
          利用双向长短期记忆网络对嵌入矩阵进行序列建模,预测下一个物品,将推荐问题转变成一个多分类问题。
    \item 考虑到循环神经网络难以并行训练的效率问题,本文抛弃神经网络循环结构,从Transformer\upcite{NIPS2017_7181}
    	  模型中抽取Self-Attention模块,%
          提出Transformer4Rec模型,该模型在获得较大%
          推荐精度的情况下还不损失性能。
\end{enumerate}
在本文实验对比所采用的两大公开数据集上,Transformer4Rec比BiLSTM4Rec的性能都表现更优,但BiLSTM4Rec的存在并不是没有意义,正是BiLSTM4Rec的出现才发现了基于循环结构的序列感知算法难以并行训练的困难,为算法的改进指明了方向。

\section{论文展望}

本文主要针对序列感知推荐领域进行了研究,利用神经网络序列建模技术,有效的提高了推荐结果的精确性,取得了一定的工作成果。但是序列感知推荐领域还存在许多的难题,这也是我们未来工作中可以持续改进的部分,主要包括以下几点:
\begin{enumerate}
    \item 本文提出的两个序列感知推荐算法主要利用了深度学习中的序列建模技术,但深度学习模型的可解释性很差,而推荐系统中又存在大量的需要可解释性场景,在后期的研究中,应使用具有可解释性的序列建模技术,例如图神经网络。
    \item 本文提出的序列感知推荐方法不需要大量的用户隐私数据与物品属性特征,但为了虽然满足某些特定的应用场景,但为了获得具有更加鲁棒性的推荐模型,应考虑将序列感知模型与现有的推荐模型结合,让不同特点的模型发挥各自的优点。
    \item 本文对基于神经网络的序列感知模型进行了粗略的设计与实现,但由于时间与条件的限制,还要许多细节没有完善,与现有的许多成熟推荐模型相比还有相当大的差距,在后期的工作中,将对本模型进行持续完善,使其能成为当今领域推荐算法工具箱中的一员。
\end{enumerate}
