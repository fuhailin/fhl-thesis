%!TEX root = ../document.tex
\begin{ack}
2016年我从湖北工业大学本科毕业,决定继续深造伊始就选择了广州这个城市,而恰好华南师范大学这所名校也选择了我,于是毫无犹豫我就开始在广州这座城市的三年学习与生活。如果当初没有选择研究生学习的话,我的人生轨迹绝对会与现在不同,也难以有现在的见识与眼界。
因此首先我要感谢自己坚持了自己的选择和为之实现付出的努力,但我还需告诫自己要脚踏实地更加努力。

一个人的命运要靠自我奋斗,也要考虑历史的进程。我读研期间正处于大数据与机器学习应用大爆发时期,行业的快速发展让我见识了许多开创性的前沿工作,也让我在毕业后能有幸进入工业界从事这方面的研究与工作,能够参与这一计算行业发展的历史进程中固然离不开国家创造的安定开放的美好局面。

当然在华师学习期间,进步的一点一滴都与这里的老师和同学们的帮助有关。进入华师的第一天,就有幸得到陈洁敏老师的指导,是她将我引入进机器学习与推荐系统研究的大门,她严格的要求和对我真挚的批评让我脚踏实地,受益至今。我的导师李建国教授,指导学生有耐心且友善和气,对学生的研究兴趣给予了充分的尊重与支持,感谢他让我能坚持自己的兴趣所在并最终做出成绩。还要感谢团队里的汤庸院长,让我能进入这个大团队一起学习,有开阔眼界的机会。另外还要特别感谢的一位老师是朱佳教授,他作为人工智能方面的学科带头人,不仅学识渊博而且对学生特别友好,亦师亦友,给与了我研究与论文写作上的重要指导,还让我能有幸参加过人工智能领域里的前沿会议,丰富了我的视野。还要感谢肖菁老师、曹阳老师开题以来给我的毕业论文提出的宝贵意见,在你们的不断提点下,我的毕业论文才得以顺利完成。我能顺利毕业离不开这几位老师的重要帮助,学生铭记于心!

实验室里一起奋斗的伙伴也是也是研究生生涯里难忘的对象,正是你们的日夜陪伴,让辛苦的研究工作也不孤单,反而还丰富多彩。祝愿毕业的师兄师姐能学习工作顺利,早日成为各领域的支柱人才,师弟师妹们也能代码无Bug,论文也漂亮。

衷心感谢我的父母付义新、赵铭对我的离家千里求学的支持!

感谢论文评审和答辩的专家教授们,感谢你们百忙之中来参与我的答辩,并且给我的论文提出了很宝贵的意见。

\end{ack}
